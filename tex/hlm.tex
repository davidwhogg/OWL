\documentclass[12pt,letterpaper]{article}

\newcommand{\foreign}[1]{\textsl{#1}}
\newcommand{\project}[1]{\textsl{#1}}

\begin{document}

\begin{abstract}
foo
\end{abstract}

\section{Rules of the photometry game}

Precise photometry is the bread-and-butter of the exoplanet community.
It is also crucial for [other things].

Imagine you are trying to do very sensitive relative photometry on a variable source.
The most precise methodologies at present involve fixing the position of the source on the detector
  (cite examples)
  to reduce dependency of the results on the flat-field or flat-fielding errors.
They also often involve defocusing the telescope
  (cite examples),
  to illuminate the pixels more uniformly
  (and hence reduce intra-pixel sensitivity issues),
  avoid saturation,
  and make the observations less atmosphere-dependent.
There are also strategies related to calibration or the choices of comparison stars
  (cite examples).
Once all these choices are made,
  the choice of methodology for actually photometering the source remains.
Most projects choose something akin to aperture photometry
  (cite examples),
  in which the pixels within an aperture are co-added with unit weights,
  and pixels outside the aperture are not used at all.
Here we ask only about this last choice:
What is the best way to measure the brightness of a nearly constant source,
  given a set of precise images of the source taken over time?

We are going to answer this question using optimization, of course;
  we are going to find the ``best'' method.
We will assume that the investigator obeyed the usual rules,
  so he or she fixed the position of the star on the detector,
  defocused the telescope,
  and avoided saturation.
However, we are also going to assume---%
  as is \emph{always the case}---%
  that there are some residual,
  unaccounted-for positional offsets (jitter) and point-spread-function changes.
These offsets and changes could be random or systematic over time,
  but we will assume that they are unknown \foreign{a priori}.
That is, the telescope house-keeping data are not good enough to track them at the level we care about.
We are trying to measure \emph{extremely} precise photometry,
  at the level of $10^{-4}$ or $10^{-5}$ or better.

The conditions we have assumed might seem strange,
  but they are generic for the most precise photometry systems currently operating.
\project{Kepler}, for instance, has all of these problems,
  as does [insert projects here, like Wright's].

% in what follows the object of photometry is a ``star'' not a ``source''.

It is possible to think of this problem as the problem of finding the best strategy in a single-player game.
\emph{The Rules} of this game are as follows:
\begin{itemize}
\item
  There is copious multi-epoch single-band imaging,
  with many epochs (many images),
  and a known position in that imaging of a star of great interest (the ``object'' star).
  There is also perhaps some ``pixel mask'' indicating what pixels are permitted for use
  in the vicinity of that star.
\item
  There may or may not be a list of ``comparison'' stars
  that are used to calibrate or ratio the photometry of the object star.
  If there aren't, then it can be assumed that the imaging is very well calibrated.
\item
  It is expected that the star is fairly stationary in device coordinates (pixel location).
  However, in each image it has been offset slightly by some amount from its fiducial position.
  There are no reliable meta-data to help you understand those small offsets.
  Any comparison stars are also subject to small offsets;
  the comparison-star offsets are not necessarily the same as the object-star offsets,
  because there is camera rotation, variable optical distortion, proper motion, and parallax.
\item
  It is expected that the point-spread function (PSF) is fairly constant.
  However, again, in each image it has a small difference from the fiducial point-spread function.
  There are no reliable meta-data to help with this,
  and the comparison star PSFs will also vary but not in precisely the same way,
  though their variations might be non-trivially correlated.
\item
  There are no reliable meta-data about read noise or gains,
  so there is no precise noise estimate for any pixel at any epoch.
\item
  When you photometer a star,
  the only operation you are permitted is linear weighted sums of pixels in the images.
  Furthermore, any weighted sums must be precisely identical from epoch to epoch.
  That is, the ``aperture''
  (which in fact may be some complicated set of pixel weights)
  is required to be identical from image to image.
\end{itemize}
These rules are generic;
  this is a common situation for precise, multi-epoch photometry.

The only rule to which we object is the last one:
In principle it should be better to do focal-plane modeling
  and perform photometry as a maximum-likelihood value
  or some statistics of a posterior PDF
  based on a parameterized likelihood function.
However, there are many applications in which
  the physical information one would like for building such a model is not available.
We would parameterize and fit nonetheless,
  but most astronomical projects don't take this approach at present.
There are also situations in which the output of full likelihood fitting for the data
  shows correlations between the final photometry and nuisance parameters (such as PSF parameters).
In these cases, the investigator is left wondering if the residual photometric variance
  has contributions from nuisance-parameter variance or fit residuals or model mismatch.

\section{Well-calibrated data}

In the (possibly fantastical) situation in which the imaging data are all properly calibrated,
  we have no need for comparison stars.
We are going do the photometry using nothing but the pixels near the object star.

Before we say what ``properly calibrated'' means,
  it is necessary to specify what we mean by an ``image''.
Ordinarily it is our view that
  an image is a measurement of the smoothed intensity field
  or smoothed photon phase-space density.
It ought to have units of energy per time per area per solid angle
  or photons per time per area per solid angle.
The problem with this view of an image
  is that if we do aperture photometry---%
  that is, if we make linear weighted sums of image pixels---%
  to get a total stellar flux,
  the weights in the weighted sum must have units of \emph{solid angle}.
Small changes to the optics or atmosphere will change the astrometric mapping of the camera,
  changing the solid angles of the pixels.
Aperture photometry is the wrong thing to do in this formulation of an image;
  or it is an okay thing to do if the weights are permitted to vary with the astrometric distortion.
That would violate \emph{The Rules} above.

So the image we want for these purposes is not a measurement of the intensity field,
  but rather a kind of ``flux map''.
It ought to have units of energy per time per area
  or photons per time per area.
These values can be co-added with dimensionless weights to deliver a stellar flux.
The only difference between an intensity image and a flux map is a factor of
  the angular size of the pixels.
In many imagers pixels are similar in size so these differences aren't apparent.
However, at the levels of precision we care about here
  ($10^{-4}$ or $10^{-5}$ or better)
  these details matter.

What is meant by ``properly calibrated''?
The image (flux map) is properly calibrated when it is can be treated as a measurement of---%
  or is directly proportional to---%
  a smoothed map of energy or photon number per time per area.
It is ``smoothed'' because it is convolved (or correlated) with some finite
  and possibly spatially varying PSF
  (which we don't know).

Are any imaging data properly calibrated according to these criteria?
We don't know, but the \project{Kepler} Satellite may be.
The standard data products of the \project{Hubble Space Telescope}
  are given in intensity units,
  but the astrometric solution is known to high precision,
  so (with a tiny bit of work)
  these also can be properly calibrated for aperture photometry according to this definition.

We do not endorse the view that images should be presented as flux maps!
We are just sayin' that \emph{if} you need to obey \emph{The Rules},
  then you need to convert your intensity maps into flux maps before playing the game.

\section{Comparison stars}

\section{Discussion}

Why, in fact, a forward-modeling approach ought to win in the end.
Why, nonetheless, this is not usually the chosen path these days.

How would this generalize to ``matched photometry'' across bands or images from different setups.

Intrapixel sensitivity variations;
  how do we know these happen and how does this method account for them?

What happens if you don't have good background subtraction or a DC level?

What happens if you have crowding or blending of stars?

\end{document}
