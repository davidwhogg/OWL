% This file is part of the OWL project.
% Copyright 2014 the authors.

% ## notes:
% - Use ``star'' not ``source'' for the target of the OWL.

% ## to-do:
% - Should I include detrendings like PDC and median filtering?
% - Needs affiliations etc.
% - Finish zeroth draft.
% - Send for comments to Lupton, Carey, Wright, Brandner, Loredo
% - Who is in charge of the PanSTARRS variability field?

\documentclass[12pt, letterpaper, preprint]{aastex}

\usepackage{color,hyperref}
\definecolor{linkcolor}{rgb}{0,0,0.5}
\hypersetup{colorlinks=true,linkcolor=linkcolor,citecolor=linkcolor,
            filecolor=linkcolor,urlcolor=linkcolor}

\newcommand{\documentname}{\textsl{Article}}
\newcommand{\project}[1]{\textsl{#1}}
\newcommand{\foreign}[1]{\textsl{#1}}
\newcommand{\etal}{\foreign{et\,al.}}
\newcommand{\transpose}[1]{{#1}^{\!\mathsf T}}
\newcommand{\inverse}[1]{{#1}^{-1}}
\newcommand{\E}[1]{\mathrm{E}[{#1}]}
\newcommand{\Var}[1]{\mathrm{Var}[{#1}]}
\newcommand{\Covar}[1]{\mathrm{Covar}[{#1}]}
\include{vc}

% affils
\newcommand{\ccpp}{1}
\newcommand{\mpia}{2}
\newcommand{\contact}{3}
\newcommand{\courant}{4}

\begin{document}\sloppy\sloppypar

\title{
  Optimized ``soft'' aperture photometry for time-domain astronomy
}
\author{
  David W. Hogg\altaffilmark{\ccpp,\mpia,\contact},
  Dan Foreman-Mackey\altaffilmark{\mpia},
  Jonathan Goodman\altaffilmark{\courant}
}
\altaffiltext{\ccpp}{%
  Center for Cosmology and Particle Physics,
  Department of Physics, New York University}
\altaffiltext{\mpia}{%
  Max-Planck-Institut f\"ur Astronomie}
\altaffiltext{\contact}{%
  \texttt{david.hogg@nyu.edu}}
\altaffiltext{\courant}{%
  Courant Institute for Mathematical Sciences,
  New York University}

\begin{abstract}
Precise stellar photometry campaigns by \project{Kepler}, \project{Spitzer}, and \project{PanSTARRS}
  (among many others)
  create precise lightcurves by fixing the star in detector coordinates,
  and estimating brightness with a fixed linear combination of background-subtracted pixel values.
Under any complete set of assumptions about the noise in the detector,
  there is a particular choice of pixel weights for use in the brightness estimate
  that maximizes signal-to-noise.
When precision requirements are extremely high,
  the noise model must necessarily include pointing and point-spread-function variations,
  which introduce highly correlated noise at the pixel level.
Here we generalize previous results on optimized photometric estimators
  to include arbitrarily correlated pixels.
Amusingly, the approach we advocate does not in the end
  require any advance knowledge of any noise source;
  once there are enough epochs observed,
  the pixel-level data are sufficient to build an empirical data-driven correlated noise model.
The optimized pixel weights can be found analytically;
  when they are used to perform photometry, they work extremely well.
We demonstrate enormous improvements to the \project{Kepler} photometry of isolated stars.
The Optimized-Weight Linear (OWL) photometry produces photometric measurements so uniform that
  they outperform or obviate much more heuristic (and much more scientifically damaging)
  ``detrending'' schemes in many real cases.
In particular, OWL photometry minimizes instrument-induced variance
  but preserves all intrinsic stellar variability.
All the OWL code is available under an open-source license.
\end{abstract}

\keywords{
  instrumentation: photometers
  ---
  methods: data analysis
  ---
  methods: statistical
  ---
  techniques: photometric
}

\section{Rules of the photometry game}

Precise photometry is the bread-and-butter of the exoplanet community.
The unprecedented precision of the \project{Kepler} photometry has led to the discovery of
dozens of Earth-radius planets (HOGG CITES).
\project{Kepler}-level (part in $10^5$) precise photometry is, no doubt,
  going to be of great importance in many other areas as well.

Imagine you are trying to do very sensitive relative photometry on a variable source.
The most precise methodologies at present involve fixing the position of the source on the detector
  (cite examples)
  to reduce dependency of the results on the flat-field or flat-fielding errors.
They also often involve defocusing the telescope
  (cite examples),
  to illuminate the pixels more uniformly
  (and hence reduce intra-pixel sensitivity issues),
  avoid saturation,
  and make the observations less atmosphere-dependent.
There are also strategies related to calibration or the choices of comparison stars
  (cite examples).
Once all these choices are made,
  the choice of methodology for actually photometering the source remains.
Most projects choose something akin to aperture photometry
  (cite examples),
  in which the pixels within an aperture are co-added with unit weights,
  and pixels outside the aperture are not used at all.
Here we ask only about this last choice:
What is the best way to measure the brightness of a nearly constant source,
  given a set of precise images of the source taken over time?

FOR THE ABOVE CHECK TWITTER FOR REFS AND THANK TWEEPS.

We are going to answer this question using optimization, of course;
  we are going to find the ``best'' method.
We will assume that the investigator obeyed the usual rules,
  so he or she fixed the position of the star on the detector,
  possibly defocused the telescope,
  and avoided saturation.
However, we are also going to assume---%
  as is \emph{always the case}---%
  that there are some residual,
  unaccounted-for positional offsets (jitter) and point-spread-function changes.
These offsets and changes could be random or systematic over time,
  but we will assume that they are unknown \foreign{a priori}.
That is, the telescope house-keeping data are not good enough to track them at the level we care about.
We are trying to measure \emph{extremely} precise photometry,
  at the level of $10^{-4}$ or $10^{-5}$ or better.

The conditions we have assumed might seem strange,
  but they are generic for the most precise photometry systems currently operating.
\project{Kepler}, for instance, has all of these problems,
  as does [insert projects here, like Wright's].

It is possible to think of this problem as the problem of finding the best strategy in a single-player game.
\emph{The Rules} of this game are as follows:
\begin{itemize}
\item
  There is copious multi-epoch single-band imaging,
  with many epochs (many images),
  and a known position in that imaging of a star of great interest (the ``object'' star).
  There is also perhaps some ``pixel mask'' indicating what pixels are permitted for use
  in the vicinity of that star.
\item
  There may or may not be a list of ``comparison'' stars
  that are used to calibrate or ratio the photometry of the object star.
  If there aren't, then it can be assumed that the imaging is very well calibrated.
\item
  The imaging is background- and foreground-subtracted,
  such that the mean level of the smooth foregrounds and backgrounds in the imaging is zero.
  The background level might vary or the subtraction of it might not be perfect,
  but we can assume that it is has a mean of zero across pixels and epochs.
\item
  It is expected that the star is fairly stationary in device coordinates (pixel location).
  However, in each image it has been offset slightly by some amount from its fiducial position.
  There are no reliable meta-data to help you understand those small offsets.
  Any comparison stars are also subject to small offsets;
  the comparison-star offsets are not necessarily the same as the object-star offsets,
  because there is camera rotation, variable optical distortion, proper motion, and parallax.
\item
  It is expected that the point-spread function (PSF) is fairly constant.
  However, again, in each image it has a small difference from the fiducial point-spread function.
  There are no reliable meta-data to help with this,
  and the comparison star PSFs will also vary but not in precisely the same way,
  though their variations might be non-trivially correlated.
\item
  There are no reliable meta-data about read noise or gains,
  so there is no precise noise estimate for any pixel at any epoch.
\item
  When you photometer a star,
  the only operation you are permitted is linear weighted sums of pixels in the images.
  Furthermore, any weighted sums must be precisely identical from epoch to epoch.
  That is, the ``aperture''
  (which in fact may be some complicated set of pixel weights)
  is required to be identical from image to image.
\end{itemize}
These rules are generic;
  this is a common situation for precise, multi-epoch photometry.

The only rule to which we object is the last one:
In principle it should be better to do focal-plane modeling
  and perform photometry as a maximum-likelihood value
  or some statistics of a posterior PDF
  based on a parameterized likelihood function (for example, \citealt{hoggwhitepaper}).
However, there are many applications in which
  the physical information one would like for building such a model is not available.
We would parameterize and fit nonetheless,
  but most astronomical projects don't take this approach at present.
There are also situations in which the output of full likelihood fitting for the data
  shows correlations between the final photometry and nuisance parameters (such as PSF parameters).
In these cases, the investigator is left wondering if the residual photometric variance
  has contributions from nuisance-parameter variance or fit residuals or model mismatch.

There have been some previous attempts to optimize photometric estimators subject to \emph{The Rules},
  in generic (but limiting) situations.
When the pointing and point-spread function are both known precisely,
  the pixel noise variance is known for every pixel,
  and the pixel noise is independent (uncorrelated) pixel-to-pixel,
  there are simple analytic results (\citealt{naylor}).
These results reduce to the psf-as-matched-filter standard practice
  in the further simplification of sky-dominated noise.
There are no previous results (to our knowledge) that take account of
  the highly covariant noise created by variable pointing or optics.
This is the first primary novelty of this work.
The second primary novelty is the realization that the noise---%
  even the covariant part created by optics and pointing---%
  can be determined empirically from the data themselves,
  with no prior knowledge of the characteristics of the detector,
  telescope, or atmosphere.

\section{Well-calibrated data}

In the (possibly fantastical) situation in which the imaging data are all properly calibrated,
  we have no need for comparison stars.
We are going do the photometry using nothing but the pixels near the object star.

Before we say what ``properly calibrated'' means,
  it is necessary to specify what we mean by an ``image''.
Ordinarily it is our view that
  an image is a measurement of the smoothed intensity field
  or smoothed photon phase-space density.
It ought to have units of energy per time per area per solid angle
  or photons per time per area per solid angle.
The problem with this view of an image
  is that if we do aperture photometry---%
  that is, if we make linear weighted sums of image pixels---%
  to get a total stellar flux,
  the weights in the weighted sum must have units of \emph{solid angle}.
Small changes to the optics or atmosphere will change the astrometric mapping of the camera,
  changing the solid angles of the pixels.
Aperture photometry is the wrong thing to do in this formulation of an image;
  or it is an okay thing to do if the weights are permitted to vary with the astrometric distortion.
That would violate \emph{The Rules} above.

So the image we want for these purposes is not a measurement of the intensity field,
  but rather a kind of ``flux map''.
It ought to have units of energy per time per area
  or photons per time per area.
These values can be co-added with dimensionless weights to deliver a stellar flux.
The only difference between an intensity image and a flux map is a factor of
  the angular size of the pixels.
In many imagers pixels are similar in size so these differences aren't apparent.
However, at the levels of precision we care about here
  ($10^{-4}$ or $10^{-5}$ or better)
  these details matter.

What is meant by ``properly calibrated''?
The image (flux map) is properly calibrated when it is can be treated as a measurement of---%
  or is directly proportional to---%
  a smoothed map of energy or photon number per time per area.
It is ``smoothed'' because it is convolved (or correlated) with some finite
  and possibly spatially varying PSF
  (which we don't know).

Are any imaging data properly calibrated according to these criteria?
We don't know, but the \project{Kepler} Satellite may be.
The standard data products of the \project{Hubble Space Telescope}
  are given in intensity units,
  but the astrometric solution is known to high precision,
  so (with a tiny bit of work)
  these also can be properly calibrated for aperture photometry according to this definition.

We do not endorse the view that images should be presented as flux maps!
We are just sayin' that \emph{if} you need to obey \emph{The Rules},
  then you need to convert your intensity maps into flux maps before playing the game.

\section{Optimized photometry}

We have at each of $N$ epochs $n$ a set of $D$ pixels in a small patch
  that contains a star.
The set of measured, properly calibrated (see above),
  correctly background-subtracted (see above)
  pixel values at epoch $n$ form a $D$-dimensional column vector $y_n$.
According to \emph{The Rules}, all we are permitted to do at epoch $n$
  is produce a weighted sum
\begin{eqnarray}
q_n &\leftarrow& \transpose{w}\, y_n
\quad ,
\end{eqnarray}
  where $q_n$ is a scalar photometric measurement,
  $w$ is a $D$-dimensional column vector of weights,
  and (because of the transpose)
  the product is effectively an inner (scalar or dot) product.
\emph{The Rules} state that the weight vector $w$ must not depend on $n$ or anything else;
  the weights are the same at every epoch.

We can see $q_n$ as an estimator---a point estimate---of the stellar brightness.
If we knew the expectation value (mean) for the pixel data $y_n$ and also the
  variance of those data around the mean,
  we could predict the expectation and variance of the estimator.
The expectation value would be:
\begin{eqnarray}
\E{q_n} &=& \transpose{w}\, \mu_n
\\
\mu_n &\equiv& \E{y_n}
\\
\Var{q_n} &=& \transpose{w}\, C_n\, w
\\
C_n &\equiv& \Covar{y_n}
\quad ,
\end{eqnarray}
  where $\E{x}$ is the expectation value of $x$,
  $\Var{x}$ is the variance of (a scalar) $x$ around its expectation value,
  and $\Covar{x}$ is the $D\times D$ symmetric, positive definite covariance matrix
  of (a vector) $x$ around its expectation value.
In the simple case of independent noise from pixel-to-pixel,
  $\Covar{y_n}$ becomes diagonal,
  and the variance $\Var{q_n}$ becomes simply the sum of the diagonal variances,
  weighted by the squared weights.
The expected squared signal-to-noise ratio $\Sigma^2$ for $q_n$ is
\begin{eqnarray}
\Sigma^2 &=& \frac{[\transpose{w}\, \mu_n]^2}{\transpose{w}\, C_n\, w}
\quad .
\end{eqnarray}

It is almost always assumed that the pixel-to-pixel noise is independent (diagonal);
  why do we not assume this here?
The answer lies in \emph{The Rules}:
Variations of the pointing and point-spread function (and background),
  which are all unknown,
  correlate the pixels.
That is, they produce a kind of noise that is neither photon nor read noise in the detector.
They produce a jitter noise that leads to covariant variations of the individual pixel values.

We win the game if we find the weight vector $w$ that optimizes signal-to-noise.
It can't optimize the signal-to-noise at epoch $n$;
  it must optimize the mean or typical signal-to-noise across epochs.
That is, we want to find the weight vector $w$ that optimizes
\begin{eqnarray}
  \Sigma^2 &=& \frac{[\transpose{w}\, \mu]^2}{\transpose{w}\, C\, w}
  \quad ,
\end{eqnarray}
where now $\mu$ is the expectation value of the data vectors $y_n$ but now over \emph{all} epochs,
  and $C$ is the covariance matrix for the data but again over \emph{all} epochs.
The signal-to-noise-optimized weight vector is simply
\begin{eqnarray}
  w &\leftarrow& \inverse{C}\, \mu
  \quad ,
\end{eqnarray}
  where the inverse operation is the full matrix inverse.
This optimized-weight result is the natural generalization to correlated noise
  of the equivalent result (for independent pixel noise)
  obtained previously (\citealt{naylor}).

Previous work has called these weights ``optimal'' but they are only really ``optimized''.
For one, the word ``optimal'' always begs the question ``with respect to what?''
For another, we reserve the word ``optimal'' for estimators
  that have some glimmer of hope of saturating the Cram\'er--Rao bound (CITE).
The optimized estimators presented here will not do this in general
  (and nor will those of \citealt{naylor}),
  because they are not based on any reasonable likelihood function.

By assumption, we don't know the point-spread function,
  nor the noise model,
  nor the pointing or PSF jitter.
Therefore, we can't compute $\mu$ and $C$ from first principles; we have to estimate them.
Fortunately, we have $N$ epochs of data;
  we can compute a (suitably robustly estimated)
  \emph{empirical} mean $\hat{\mu}$ and \emph{empirical} covariance $\hat{C}$
  of the original data vectors $y_n$.
That is, we will do something like
\begin{eqnarray}
  \hat{\mu} &\leftarrow& \frac{1}{N}\,\sum_{n=1}^N y_n
  \\
  \hat{C} &\leftarrow& \frac{1}{N-1}\,\sum_{n=1}^N [y_n - \hat{\mu}]\,\transpose{[y_n - \hat{\mu}]}
  \\
  \hat{\Sigma}^2 &=& \frac{[\transpose{w}\, \hat{\mu}]^2}{\transpose{w}\, \hat{C}\, w}
  \\
  \hat{w} &\leftarrow& \inverse{\hat{C}}\, \hat{\mu}
  \quad ,
\end{eqnarray}
  where the operations are averages over the epochs,
  the vector product is the outer or tensor product,
  $\hat{\Sigma}^2$ is the \emph{empirical} signal-to-noise,
  and $\hat{w}$ are the weights that maximize it.
Technically we require $N \gg D$
  (far more epochs than there are pixels in the tiny patch).
It turns out that with small modifications
  (to make the means robust to outliers)
  this works well, at least for \project{Kepler} data (see below).
The Optimized-Weight Linear (OWL) photometry $q_n$ is then
\begin{eqnarray}
  q_n &\leftarrow& \transpose{\hat{w}}\, y_n
  \quad ,
\end{eqnarray}
  that is, one scalar $q_n$ per epoch $n$, obtained by a weighted sum of the $y_n$.
We have not just obeyed \emph{The Rules},
  we have come up with the best possible strategy in terms of empirical signal-to-noise.

There is one overall scale that is free;
  any scalar multiple of the weight vector $\hat{w}$ is identical in signal-to-noise
  to any other scalar multiple.
In practice it doesn't matter how we choose this unless either machine precision is an issue,
  or else some kind of absolute aspects of photometry are required.

\section{\project{Kepler} data}

For each star in the Kepler data (identified by a KIC number),
  there is a small patch of telemetered pixels,
  and some 4000 epochs per quarter and up to 17 quarters of data.
The patch of pixels has an associated mask,
  which indicates both which pixels are good data,
  and also which were summed to make the official pipeline
  Simple Aperture Photometry (SAP).
The weights used to generate the SAP photometry are constant within a quarter,
  and all snapped to either zero or unity.
\figurename~\ref{fig:pixels} shows time histories for the 19 pixels relevant
  to KIC~3335426 from the fifth quarter in which it was observed.
Also shown are the three pixels that got unit weight in the SAP photometry for this star.

The \project{Kepler} data are extremely good, but there are some outliers,
  presumably from cosmic rays.
We want to compute empirical means and covariances that are not overly drawn by these outliers.
We compute the empirical statistics with an iterated sigma-clipping.
In each iteration, we censor (remove from the sum)
  any datum $y_n$ that is more than a ``4 sigma'' outlier
  given the empirical mean $\hat{\mu}$ and covariance $\hat{C}$ computed in the previous iteration.
To determine the 4-sigma criterion, we use the Gaussian approximation to the chi-squared distribution
  with the appropriate number of degrees of freedom.
We iterate this sigma-clipping through five iterations,
  which in practice seems to lead to convergence.
Importantly, we sigma-clip at the epoch level, not the pixel level.

\figurename~\ref{fig:images} shows the robustly estimated empirical mean image $\hat{\mu}$,
  the diagonal elements of the empirical covariance matrix $\hat{C}$,
  the top eigenvector of the covariance matrix,
  and the OWL weights, along with the weights used by the \project{Kepler} pipeline,
  and a few data examples.
We resolved the overall scale degeneracy for the weights (see above)
  by requiring that the OWL and SAP photometry produce identical median flux.
The optimized weight vector shown in \figurename~\ref{fig:images} might seem counterintuitive;
  it puts substantial weight off the peak of the mean intensity.
The fundamental reason is that the weight vector is trying to get as orthogonal as possible
  to the high-variance eigenvectors of the variance tensor.
In the end, the highest-weight pixels do not all deliver significant signal;
  in \figurename~\ref{fig:images} we also show
  the pixel-wise product of the OWL weights with the empirical mean values,
  which gives a sense of where the bulk of the photometric signal is coming from.

\figurename~\ref{fig:results} shows the \project{Kepler} SAP flux for KIC~3335426,
  along with the OWL and some variants thereof (described below).
The OWL photometry very effectively  detrends the data,
  even though it involves \emph{no explicit detrending}.
By ``detrending'' we mean filtering or fitting-and-subtracting
  the SAP photometry (as in the \project{Kepler} PDC or median-filtered photometry).
Detrendings remove intrinsic stellar variability and create artificial epoch-to-epoch correlations
  in the photometric noise;
  they are the bane of exoplaneteers.
That is, the OWL is just a soft-aperture aperture photometry, identical from epoch to epoch
  that obviates detrending.
It is simply the weighting of the pixels that optimizes the empirical output signal-to-noise.

In detail, this first shot at OWL photometry is a tiny bit noisier
  in an epoch-to-epoch sense
  than the SAP photometry,
  despite being far better in its overall trend.
This is coming from the fact that the OWL weights are trying to suppress the dominant
  sources of variance,
  which are entering the data on long time-scales.
The focus on the biggest variance components limits what the OWL can do.
This opens up the next area of discussion:
How do we estimate the covariance matrix $\hat{C}$ so that it is dominated by the sources of variance
  that we most want to suppress?

\section{Estimation of the covariance tensor}

HOGG:
Something about what is really happening (trying to zero out dominant eigenvectors).

HOGG:
Something about the lack of prior knowledge:
\begin{itemize}
\item
We aren't assuming \emph{anything} about the sources of variance.
\item
We aren't doing \emph{anything} about intrinsic variability of the source.
\item
Some intrinsic variability we do care about, some we don't.
\item
We aren't using the time-ordering of the data at all;
  different effects happen on different time scales.
\end{itemize}
All these issues turn into decisions or knobs that we can turn in variations.
The variations will fall into two categories:
We can change how we estimate the covariance matrix $C$,
  or we can change how we constrain or regularize the weights $w$.

On the covariance estimation, here are some basic ideas:
\begin{itemize}
\item
Use local differences of pixel values instead of pixel values.
\item
Mix the dominant eigenvectors from the full set of pixel values
  with the small-scale noise inferred from the local differences.
\item
Perform some kind of factor analysis that splits the variance into
  ``diagonal'' and ``covariant'' parts.
\item
Use covariances across pixel patches from different stars
  to identify extrinsic (rather than intrinsic)
  sources of variances.
\end{itemize}

HOGG:
In yet another variation, we estimated $\hat{\mu}$ and $\hat{C}$ using not the
  straight-up data, but the \emph{differences} between each data point and
  its time-ordered neighbor.
Again with the sigma clipping.
This leads to the DOWL, DOPW, and DTSA, shown in \figurename~\ref{fig:dresults}.

\section{Regularization of the weight vector}

It is also the case, in the above, that
  we aren't putting in any prior knowledge about what the $w$ image ought to look like
  in sane cases.
On the weight constraints and regularization, here are some basic ideas:
\begin{itemize}
\item
Require non-negativity.
\item
Require smoothness.
\item
Require similarity to the SAP aperture or some other sensible aperture.
\item
Require that the aperture be a circular top-hat or a smoothed version of that.
\end{itemize}

After running the OWL on multiple \project{Kepler} targets,
  we developed a few intuitions that led to variations.
In one, we constrain the weights to be positive;
  this protects against the optimizer finding locations in weight-space
  that perform subtractions of large numbers to reduce variance.
This problem arises most strongly when the star being photometerd is intrisically variable
  with a large amplitude:
In these cases, the largest variance can be from the star itself;
  the OWL tries to find weights that cancel out the target star!

The Optimized Positive-Weight (OPW) photometry works still at the optimum of
\begin{eqnarray}
\hat{\Sigma}^2 &=& \frac{[\transpose{w}\, \hat{\mu}]^2}{\transpose{w}\, \hat{C}\, w}
\quad ,
\end{eqnarray}
but now subject to the constraint that every element of $w$ must be non-negative.
\figurename~\ref{fig:images_opw} and \figurename~\ref{fig:results}
 show the results of the OPW for KIC~3335426,
 and \figurename~\ref{fig:results_variable} show the results of the OPW
 for a more variable star.
The positivity constraint seems salutory when the target star is somewhat variable.

In another variation, we looked at stars that saturate the \project{Kepler} detectors.
These stars show bleed trails along the CCD columns,
  although the \project{Kepler} gains have been set cleverly
  to ensure that precise photometry of these saturated stars is nonetheless possible with SAP photometry.
In the case of saturated stars, the OWL and OPW photometry do not work well
  because the saturation leads to some pixels being exceedingly constant,
  even if the source is intrinsically moving or varying.
That is, the saturated pixels do not show much empirical variance,
  and get overly large weight in the optimization.

For these saturated cases we devised a Tweaked Simple Aperture (TSA) photometry,
  which capitalizes on the clever apertures chosen by the \project{Kepler} team
  to capture saturated signal:
In the TSA, we lump together the SAP-coadded pixels into a single super-pixel.
We then perform the OWL methodology,
  but now with the list of pixels not being the full list of pixels,
  but rather the super-pixel plus the list of all the non-SAP pixels in the patch.
The TSA is a generalization of the OWL to non-square or non-uniform pixels.

We show the TSA results in \figurename~\ref{fig:results}.
It seems to be producing better epoch-to-epoch scatter;
  this might because it represents an even stronger regularization of the weight image $\hat{w}$.

HOGG:
In yet another variation, we fix the aperture to be a circular top-hat,
  and only permit its center and radius to be adjusted in the optimization...

NOTE TO JASON WRIGHT: WE CAN HELP YOU!

\section{Comparison stars}

HOGG:  GIVE THE SETUP

HOGG:  You can operate the OWL for all $K$ stars.
Then scale each star so it predicts as well as possible the object star.
Then take robust means or medians of the $K$ predictors and divide them out.

HOGG:  Or you can use all of the pixels in all the patches around all $K$ stars
  to build a predictive model for the pixels in the object star's patch.
This permits creation of a very stable data-driven calibration of the data.
We are pursuing this in a separate project;
  it is not even close to being ready for prime time.

\section{Discussion}

What happened?
We permitted the aperture to be soft.
We optimized the aperture weighting.
We obviated detrending forever.

In the results shown in \figurename~\ref{fig:results},
  the instrumental effects on the lightcurve were removed beautifully by the OWL,
  but in detail
  the adjacent epoch-to-epoch noise was inflated a tiny bit;
  why?
The OWL knows nothing about time ordering of the data;
  it makes no distinctions between slow and fast variations;
  it is just trying to maximize signal-to-noise.
Slow variations like those shown in \figurename~\ref{fig:results}
  are just another source of total variance.
If a user knows a lot about what timescales are relevant,
  the OWL can be improved enormously.
To demonstrate, we ran an experiment in which we filtered the \project{Kepler} pixel-level data
  for KIC~XXXX in quarter XXXX
  with a 48-hour median filter to remove the long-timescale variability.
We subtracted the median filter output from each pixel lightcurve and added back in the mean.
The pixel lightcurves are shown in \figurename~\ref{fig:pixels_filter}
  and the output of the OWL photometry is shown in \figurename~\ref{fig:results_filter}.
Because the pixels were partially ``detrended'' by this process,
  the OWL weights are not putting as much effort into removing the large-scale trends.

This kind of detrending is common in the exoplanet community;
  it has been valuable for finding exoplanets.
It brings many disadvantages;
  one is that any detrending method that happens prior to exoplanet search
  distorts the transit shapes and depths.
Another is that it removes stellar variability that might be of intrinsic interest.
Another is that it correlates data that are close in time;
  it creates non-trivial noise for lightcurve fitting.
The OWL photometry---in its standard form shown in \figurename~\ref{fig:results}---%
  produces independent photometry at every epoch
  and preserves transit shapes and all intrinsic stellar variability.

In the end, a forward-modeling approach ought to win.
It won't obey \emph{The Rules}, but it is the only approach that has a chance
  of saturating the Cram\'er--Rao bound.
The forward-modeling approach is not trivial, however,
  because the usual methods of image modeling,
  which make use of simple models for the flat-field, point-spread function,
  and the ``scene''
  have (in general) model mismatch issues that project residuals
  in the calibration parameters onto the photometry of greatest interest.
That is, a forward-modeling approach that obtains photometry at the same
  precision as the OWL would be expensive and complex.
Still, it is the \emph{Right Thing To Do (tm)} and thus the One True Path for the future.
One way to see this is to note that there is no obvious or simple generalization of the OWL
  to the case of multi-instrument or multi-band imaging
  (where pixel scale or bandpass vary from epoch to epoch).

The OWL, OPW, and TSA methods all depend very strongly on good background subtraction.
The reason is that it seeks to obtain high expectation value and low variance.
If there is a constant background that hasn't been subtracted,
  the weights will choose to add in a lot of constant background
  and possibly down-weight the pixels being hit by the variable object star.
The main requirement is that the background has been subtracted such
  that the mean background level in all pixels at all epochs is negligible.
The background subtraction can be noisy and variable,
  but it must be accurate in the mean.

On a related note, the image patch under consideration ought not be \emph{far}
  bigger than the region over which a normal would perform photometry.
For example, if the number of pixels in the object-star image patch becomes
  much larger than the number of epochs,
  the OWL will be able to find linear combinations of pixels that have
  more-or-less perfectly constant flux.
This over-fitting issue is not a problem in what's shown above,
  but could start to become a problem if large patches get used,
  and would be exacerbated in the presence of non-zero mean background subtraction residuals.
This deserves further study.

The biggest (in our view) limitation of the OWL is that it cannot be used
  to accurately measure the properties of stars
  that are close enough that their point-spread functions overlap,
  or stars in crowded fields.
The issue is that the optimization procedure has no knowledge or understanding
  of the PSF or even of the fact that it is a star being observed.
In our view, the only justifiable solution to this problem is forward modeling,
  in which there are parameterized beliefs about the PSF and the scene,
  and in which the output photometry comes from analysis of a likelihood function.
The only conceivable generalizations of the OWL that might be relevant would either
  \textsl{(a)}~involve taking the optimum subject to the constraint that the weights must be
  orthogonal to some estimate of the overlapping star point-spread functions, or else
  \textsl{(b)}~require modeling and subtraction of the confusing source or sources
  in the field.
These are interesting ideas,
  but probably no easier or more reliable than full forward modeling.
Although we consider this non-isolated problem to be the most severe limitation of what we are doing,
  we note that the tremendous discoveries and measurements of the \project{Kepler} mission
  have almost all been generated with straight-up aperture photometry,
  followed by \foreign{ex post facto} checks for blending.
There certainly is a great deal that can be done in the isolated-star limit.

All of the code used in this project is available
  from \url{http://github.com/davidwhogg/OWL/} (or forks thereof)
  under the MIT open-source software license.
This code (plus some dependencies) can be run
  to re-generate all of the figures and results in this \documentname;
  this version of the paper was generated with git hash
  \texttt{\githash} (\gitdate).

\acknowledgments
It is a pleasure to thank
  Ruth Angus (Oxford),
  Morgan Fouesneau (MPIA),
  Fengji Hou (NYU), 
  Christian Knigge (Southampton),
  Molei Tao (NYU), and
  Jason Wright (PSU)
for valuable comments and suggestions.
This work was partially supported by the Moore--Sloan Data Science Environment at NYU,
  by NASA (grant NNX08AJ48G).
  and by the NSF (grant AST-0908357).
This research made use of the \project{NASA Astrophysics Data System}.

\newcommand{\arxiv}[1]{\href{http://arxiv.org/abs/#1}{arXiv:#1}}
\begin{thebibliography}{}\raggedright
\bibitem[Hogg \etal(2013)]{hoggwhitepaper}
Hogg, D.~W., Angus, R., Barclay, T., et al.\ 2013,
Maximizing Kepler science return per telemetered pixel: Detailed models of the focal plane in the two-wheel era,
\arxiv{1309.0653}
\bibitem[Naylor(1998)]{naylor}
Naylor, T.\ 1998,
An optimal extraction algorithm for imaging photometry,
\mnras, 296, 339
\end{thebibliography}

\clearpage
\begin{figure}
\includegraphics[width=\textwidth]{../py/kic_03335426_05_pixels.png}
\caption{
The ``lightcurves'' (flux \foreign{vs} time) for 19 pixels near...HOGG
\label{fig:pixels}}
\end{figure}

\clearpage
\begin{figure}
\includegraphics[width=\textwidth]{../py/kic_03335426_05_images_owl.png}
\caption{
The top row shows example image patches (data $y_n$) from the \project{Kepler} data on KIC XXXX in quarter XXXX.
The middle row shows the empirical mean image $\hat{\mu}$,
  the diagonal elements of the empirical covariance matrix $\hat{C}$,
  and its dominant eigenvector.
The bottom row shows the pixel weights used in the official \project{Kepler} SAP photometry,
  and the OWL weights $\hat{w}$.
The bottom-right panels shows the OWL weights times the empirical mean;
  this shows the mean contributions of the individual pixels to the final OWL photometry.
\label{fig:images}}
\end{figure}

\clearpage
\begin{figure}
\includegraphics[width=\textwidth]{../py/kic_03335426_05_images_opw.png}
\caption{
The same as the bottom row of \figurename~\ref{fig:images}
  but for the optimized positive-weight (OPW) photometry.
In this case, the positivity constraint does not make much difference.
\label{fig:images_opw}}
\end{figure}

\clearpage
\begin{figure}
\includegraphics[width=\textwidth]{../py/kic_03335426_05_photometry.png}
\caption{
The \project{Kepler} SAP flux,
  the Optimized-Weight Linear (OWL) photometry,
  the Optimized Positive-Weight (OPW) photometry,
  and the Tweaked Simple Aperture (TSA) photometry,
  all for quarter 5 of KIC~3335426.
\label{fig:results}}
\end{figure}

\clearpage
\begin{figure}
\includegraphics[width=\textwidth]{../py/kic_03335426_05_diff_photometry.png}
\caption{
The \project{Kepler} SAP flux,
  the Difference-Optimized-Weight Linear (DOWL) photometry,
  the Difference-Optimized Positive-Weight (DOPW) photometry,
  and the Difference-Tweaked Simple Aperture (DTSA) photometry,
  all for quarter 5 of KIC~3335426.
\label{fig:dresults}}
\end{figure}

\end{document}
